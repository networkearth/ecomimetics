\documentclass[11pt,a5paper]{book}
\usepackage[utf8]{inputenc}
\usepackage{amsmath}
\usepackage{amsfonts}
\usepackage{amssymb}
\usepackage{graphicx}
\usepackage[super]{nth}

\title{Learning to Coexist}
\author{Marcel Gietzmann-Sanders}
\date{}
\setcounter{tocdepth}{1}
\begin{document}
\maketitle
\tableofcontents
\newpage
\chapter{What Is this Book?}

When you really sit down and think about it, the amount of diversity in the world around us is legitimately very surprising. While we often like to portray nature as this harmonious place, the fact of the matter is that it is full of strife, competition, predation, disease, resource limitations, and so on. And as many early ecologists realized, if you have a bunch of creatures naively competing for the same resources, whoever is the best at it just wins, and the rest disappear.

Yet, here we are, in a world where there are more species than we know how to count — all coexisting. Given our own questions about ecological stewardship more or less boil down to questions about coexistence between ourselves and the creatures we share this how with; rather than trying to reinvent the wheel, why not learn from the master herself? \newline

As such, this book is nothing more than an attempt to collect and understand the tools and strategies by which Mother Nature has achieved this extraordinary feat. If we can use the tools that nature has used to create the diversity we see around us, I think we stand a pretty good chance of learning how to properly coexist ourselves.

But to do so, we first have to understand those tools, and, as I always say, you only truly understand something when you can teach it to someone else. So... onto the zoo. 

\part{The Zoo}

\end{document}