\documentclass[11pt,a5paper]{book}
\usepackage[utf8]{inputenc}
\usepackage{amsmath}
\usepackage{amsfonts}
\usepackage{amssymb}
\usepackage{graphicx}
\usepackage[super]{nth}

\title{Learning to Coexist}
\author{Marcel Gietzmann-Sanders}
\date{}
\setcounter{tocdepth}{1}
\begin{document}
\maketitle
\tableofcontents
\newpage
\chapter{What Is this Book?}

When you really sit down and think about it, the amount of diversity in the world around us is legitimately very surprising. While we often like to portray nature as this harmonious place, the fact of the matter is that it is full of strife, competition, predation, disease, resource limitations, and so on. And as many early ecologists realized, if you have a bunch of creatures naively competing for the same resources, whoever is the best at it just wins, and the rest disappear.

Yet, here we are, in a world where there are more species than we know how to count — all coexisting. Given our own questions about ecological stewardship more or less boil down to questions about coexistence between ourselves and the creatures we share this how with; rather than trying to reinvent the wheel, why not learn from the master herself? \newline

As such, this book is nothing more than an attempt to collect and understand the tools and strategies by which Mother Nature has achieved this extraordinary feat. If we can use the tools that nature has used to create the diversity we see around us, I think we stand a pretty good chance of learning how to properly coexist ourselves.

But to do so, we first have to understand those tools, and, as I always say, you only truly understand something when you can teach it to someone else. So... onto the zoo. 

\part{The Zoo}

\chapter{Playing with Toys}

One of the single best ways to understand how something works is to try to build it yourself from scratch. It's a great way to distill the elements that matter from those that are just noise. Continuing down this train of thought, while it may be tempting to try to outline a process in all of its complexity all at once, you can get a lot of mileage by playing a kind of "straw-man" game. That is you start with a model that is clearly too simple and then use it to figure out exactly \textit{why} it is too simple. This more or less prevents you from sticking in more detail than you actually need because many times even the simplest of models can surprise us and more often than that the complexities we think we need are not the right ones at all. 
\newline

All to say that in this chapter we're going to start with some models that are excruciatingly simple in that they can barely be said to model anything at all. Yet as I hope you'll see we're actually going to get quite a bit of mileage out of them in terms of understand what kinds of mechanisms are really at play in the world around us. 
\newline

Let's open the toy chest. 

\section{Consumer Resource Models}

We start with a simple question - why don't consumers consume all of the resources available to them? We begin with our resource $R$ (maybe some kind of plant our consumer will feed on) and propose a very simple model for it's growth on it's own:

$$\frac{dR}{dt} = r(1-R/K)R$$

where we designate $K$ the carrying capacity of the resource $R$ and $r$ the nominal growth rate. All this model says is that the growth rate is proportional to the size of $R$ where the proportion decreases as $R\rightarrow K$. If $R$ is very small the growth is slow because there are not many individuals breeding, if $R$ is close to $K$ there is not much growth because our plants are themselves hitting some wall (maybe space, nutrients, or light). 

So, left to its own devices this is how our resource would develop. But what about the consumer?

In this case we're going to hinge our model on how the consumer should consume at the low and high ends of resource availability. So, if our resource is really low "each" consumer will end up with little to nothing, but if the resource is exceptionally high the individual consumer's consumption rate should flatten out at some maximum value (if I give you ten pounds of potatoes you're not going to eat all ten pounds). We can model this by saying that the rate at which our resource is consumed per individual unit of consumer is given by:

$$\frac{zR}{R+S}$$

There is however one problem with a model of consumption per unit consumer like this one - and that is it has no relation to the density of the consumers itself. Obviously if the resource $R$ is held fixed and we increase the consumer $C$ indefinitely eventually competition must take hold and the consumption per consumer must drop. Therefore instead of the model above we'll base on off of fullness $F=C/R$ like the following:

$$\frac{z}{1+\alpha F}$$

where now $\alpha$ is replacing $S$ as the control of how quickly saturation can happen and we can now see that as $C$ increases the rate drops as well (if $R$ is held constant). 

Putting this all together we now have:

$$\frac{dR}{dt} = r(1-R/K)R-\frac{z}{1+\alpha F}C$$

Now assuming that our consumers turn resource into biomass at a rate of $b$ we have:

$$\frac{dC}{dt}=b\frac{z}{1+\alpha F}C$$

This of course is senseless because it means a consumer, once created can never die, so let's go ahead and add some kind of base metabolism $m$ that has to be maintained per consumer:

$$\frac{dC}{dt}=b\frac{z}{1+\alpha F}C-mC=\left(\frac{bz}{1+\alpha F} - m\right)C$$

Amazing! Now we can make some observations of this model. 

\subsection{When There's Trouble in Heaven}

First, what happens when $R$ starts getting very low (and we start getting worried)? Specifically in this case we find that the $(1-R/K) \approx 1$ and therefore:

$$\frac{dR}{dt} = rR-\frac{z}{1+\alpha F}C$$

Expanding the consumption term we have:

$$\frac{dR}{dt} = rR-\frac{z}{1+\alpha C/R}C$$

So as $C\rightarrow \infty$ we get:


$$\frac{dR}{dt} = rR-\frac{z}{\alpha}R=\left(r - \frac{z}{\alpha}\right)R$$

Which is to say that at high concentrations of our consumer our resource will grow (or more likely shrink) at a rate of $r-z/\alpha$ per unit resource. What happens to $C$ in this realm? 

$$\frac{dC}{dt}=\left( b\frac{z}{1+\alpha F}-m \right)C$$

Well given that $F\rightarrow \infty$ as $C \rightarrow \infty$ and $R\rightarrow 0$ it'll become:

$$\frac{dC}{dt}=\left( 0-m \right)C$$

So while the resource is changing at $r-z/\alpha$ per unit $R$, $C$ is changing at $-m$ per unit $C$. Put another way our fullness after a short timestep $\delta t$ will be approximately:

$$F_{t+\delta t} \approx \frac{1-m\delta t}{1 + (r-z/\alpha)\delta t}\frac{C_t}{R_t}$$

We can actually learn a lot from this. Before doing so let's call $\gamma=z/\alpha-r$ the saturation consumption rate in the sense that it is how much $R$ shrinks when $C$ is very very large and $R$ is very very small (when it happens to have its highest growth rate). 

With that out of the way we have:

$$F_{t+\delta t} \approx \frac{1-m\delta t}{1 - \gamma \delta t}\frac{C_t}{R_t}$$

We can see here that if the $|\gamma|$ is larger than $|m|$ the fullness will just increase as $C$ is falling more slowly than $R$ is falling. This will mean that our situation will become more and more saturated (from a fullness $F$ perspective) and as a result the resource will get killed off by the consumer and subsequently the consumer will starve itself. 

If on the other hand $|\gamma| < |m|$ then the consumer will die off faster than the resource and the fullness will begin to drop as well taking pressure off of the resource. This will continue until $C$ and $R$ find some balance in their growth rates (or oscillate around that balance) but the fullness will never be allowed to walk off to $\infty$. 

It might seem as if we are all good now but remember this just tells us that $F$ can only get so high. It does not tell us how low $C$ and $R$ can get. For all we know a perfectly suitable solution in this $F$ bounded situation is that $C$ and $R$ both walk down to oblivion together. 
\newline

However we do know this: \textbf{if $|\gamma| > |m|$ there exists a fullness $F>>1$ s.t. the resource cannot recover.}

Specifically we know that fullness is the one where the loss rate per unit $R$ exceeds the loss rate per unit $C$. Put in more english terms - \textbf{if the consumer can remove the resource faster than it itself dies, the resource, and thereby the consumer, is doomed.} 
\newline 

So what of our other case? The one where the fullness $F$ may not necessarily decrease but the $C$ and $R$ drive one another to ruin? In this situation there are two possible cases:

\begin{enumerate}
\item $C$ and $R$ drop at precisely the same rate (on average).
\item $C$ drops off faster than $R$. 
\end{enumerate}

Given the former case is incredibly unlikely (it would require a perfect balance right until the end) we will focus on the second. In the case where $C$ drops off faster than $R$ (proportionately) $F\rightarrow 0$ which means our growth rate $dR/dt\rightarrow r$. In other words while the resource $R$ can be suppressed to very low levels eventually it must start growing again and recommence the cycle.
\newline

Now there are certainly parts of the space where $C$ has a larger effect on $R$ than it does on itself. For example if $R=K$, then the resource is producing no growth at all and so $R$ can only drop. If $C$ is small enough so that it is still growing then clearly $C$ is having more of an effect on $R$ than on itself. However in this case as $R$ drops it will begin to have its own growth rate rise and therefore can counter $C$'s effect. The reason we focused on the case where $R\rightarrow 0$ is because here the resource has no more options. \newline

We can summarize our findings as follows:

\begin{enumerate}
\item If $|\gamma| > |m|$ then, given $R$, there exists a fullness $F>>1$ s.t. the resource cannot recover.
\item If $|\gamma| < |m|$ then the consumer cannot totally annihilate the resource. 
\end{enumerate}

\subsection{Safe Zones}

So we know that if $|\gamma| > |m|$ there's potential for problems, but are there also portions of the space we don't have to worry about? Trivially, yes, and we can see this by computing when $dR/dt$ and $dC/dt$ are both zero (i.e. when there's no growth or loss). 

$$0 = \left( b\frac{z}{1+\alpha F}-m \right)C \rightarrow F= \frac{zb-m}{\alpha m}$$

$$0 = r(1-R/K)R-\frac{z}{1+\alpha F}C \rightarrow r(1-R/K)=\frac{zF}{1+\alpha F}$$

$$\rightarrow R = K\left( 1 - \frac{z}{r}\frac{F}{1+\alpha F} \right)$$

If $zb-m>0$ then there exists a solution where $C>0$ and then if the term in parenthesis in the last equation is also $>0$ we have an $R$ as well. Both of these conditions can be thought of as viability conditions. 

\textbf{TODO: Do better by the viability criteria above.}

\textbf{TODO: How large an attractor exists around the stable point?}





\end{document}